\section{Zusammenfassung und Ausblick}

	Die derzeitige Versorgungslage der fossilen Energieträger hat sich mit Hinblick auf die Kraftwerksreserven erst einmal entspannt. 
	Die Braunkohlekraftwerke aus der Sicherheitsbereitschaft können über werkseigene Tagebaue und Schienensysteme versorgt werden. 
	Die Pegelstände der Versorgungsrouten für die Steinkohlekraftwerke haben sich ebenfalls erholt. 
	Vereinzelt können Steinkohlekraftwerke jedoch nicht am Markt teilnehmen aufgrund von mangelndem Personals und Alter der Kraftwerke. 
	Zudem wurden, wie am Beispiel Saarland gezeigt, Verordnungen erlassen, um eine Brennstoffbeschaffung zu vereinfachen. 
	Die angespannteste Versorgungslage zeigt derzeit das Erdgas auf. 
	Trotz der geplanten Inbetriebnahme zweier FSRU’s Anfang Dezember und sehr gut gefüllten Gasspeichern kann keine Entwarnung gegeben werden. 
	Derzeit profitiert Deutschland von milden Wintertagen und einem daraus reduzierten Gasverbrauch sowie weiteren Einsparmaßnahmen in der Industrie. 
	Die Versorgungslage gestaltet sich damit äußerst dynamisch und ist schwer vorherzusagen. 
	Zusätzlich wurde ein von der Bundesregierung in Auftrag gegebener Stresstest angefertigt, um potenzielle Versorgungslücken aufzuzeigen und rechtzeitig Gegenmaßnahmen einzuleiten. \\
	
	Zusammenfassend sei noch einmal gesagt, dass beide Studien mit unterschiedlichen Ansätzen ähnliche Ergebnisse liefern. 
	Es wird ein weitreichender Ausbau von erneuerbaren Energien, Batteriespeicherkapazitäten, \Htwo-ready Gaskraftwerken und dem damit verbundenen Netzausbau bzw. -modernisierung. \\
	
	
	Aufgrund der genannten Kritikpunkte ist das Erreichen des Ausbauziels kaum bzw. nicht erreichbar. Hinzukommen weitere Probleme wie Materialengpässe und Personalmangel.
	Zudem kann ohne zusätzliche Gesetzesänderungen von ca. fünf Jahren Planungs- und Bauzeit ausgegangen werden (Planung, Bau und Inbetriebnahme).
	Demnach scheint nicht nur der zeitliche und personelle, sondern auch der gesetzliche Rahmen für das Erreichen der Ausbauziele nicht vollumfänglich gegeben.
	Folglich werden Reserven aus Kohlekraftwerken auch in ferner Zukunft weiter bestehen müssen, auch wenn die Wirtschaftlichkeit durch mangelnde Betriebsstunden auch in diesem Fall ungeklärt bleibt.
	
	
	
	