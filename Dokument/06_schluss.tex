\section{Zusammenfassung und Ausblick}

	Im ersten Schritt ist festzustellen, dass beide Studien ähnliche Ergebnisse liefern.
	Zukünftige Maßnahmen zur Frequenzstabilisierung und zum Redispatch sollen vorrangig mit Gas- sowie Wasserkraftwerken abgedeckt werden.
	Zu Beginn werden die Gaskraftwerke mit Erdgas betrieben, um diese anschließend sukzessive auf grünen Wasserstoff umzurüsten.
	Für die zusätzliche Überbrückung von Dunkelflauten müssen weitere \Htwo--ready Gaskraftwerke mit einer Gesamtleistung von etwa \SI{28}{\giga\watt} - \SI{30}{\giga\watt} bis 2030 zugebaut werden.
	Durch bilaterale flexible Verbraucher sollen auftretende Lastspitzen in der Erzeugung sowie Verbrauch ausgeglichen werden.
	Zudem soll so die Abschaltung von erneuerbaren Energien bei Erzeugungsspitzen verringert werden.
	Zu den flexiblen Verbrauchern zählen unter anderem Wärmepumpen, Elektroautos, Energie- sowie Wärmespeicher und Power-to-X Anwendungen.
	All diese Techniken vereinen die Möglichkeit, schnell und flexibel angefahren werden zu können und damit Last- sowie Erzeugungsspitzen zu dämpfen. \\
	
	Außerdem ist darauf hinzuweisen, dass im Zuge des Kohleausstiegs die Netzreserve und Sicherheitsbereitschaft zur Reserveleistungsvorhaltung wegfallen werden.
	Der damit verbundene Wegfall von Momentanreserve kann durch eine erhöhte Leistungsvorhaltung in der Primärregelreserve kompensiert werden.
	Damit rücken gerade großflächig angelegte Batteriespeicher in den Blickpunkt.
	Diese können sehr schnell große Leistungen abrufen und stellen damit ein erhebliches Potenzial dar. 
	Zudem kann eine weitere Systemdienstleistung, welche schneller als die Primärregelreserve eingreift, Abhilfe  schaffen.
	In dieser sind dann Teilnehmer gebündelt, die extrem schnell in den Markt eingreifen können und vor allem damit die Frequenz stabilisieren können. \\
	
	Um den in den Studien geplanten Zubau zu erreichen, werden jedoch kaum Anreize geliefert.
	Aufgrund des fortschreitenden Ausbaus der Erneuerbaren werden die Betriebsstunden der dringend gebrauchten Gaskraftwerke weiter reduziert werden (vgl. Kap. \ref{sect: Wie funktioniert der deutsche Strommarkt?}).
	Durch mangelnde Betriebsstunden wird der Betrieb solcher Grenzkostenkraftwerke immer unwirtschaftlicher.
	Folglich wird der Bau dadurch immer unwahrscheinlicher.
	Um diesen Effekten entgegenzuwirken, muss die vorgehaltene Leistung, auch wenn diese nicht abgerufen wird, vergütet werden.
	Der Strommarkt würde weiterhin nach einer Merit-Order funktionieren und erbrachte elektrische Arbeit vergüten, jedoch durch einen Kapazitätsmarkt auch vorgehaltene Kraftwerksleistung. \\
	
	Des Weiteren besitzt Deutschland noch keine funktionierende Wasserstoffwirtschaft. 
	Da der gesamte Bedarf an grünem Wasserstoff nicht durch nationale Produktion gedeckt werden kann, muss die Differenz aus anderen Ländern importiert werden.
	Hierfür gibt es jedoch keine massentauglichen Transportmöglichkeiten bzw. Lieferverträge und Umschlagplätze.
	Eine Umrüstung der geplanten und bereits gebauten LNG-Terminals ist ohne etwaige größere Investitionen nicht problemlos möglich \cite{Frauenhofer_LNG}. \\
	
	Aufgrund der genannten Kritikpunkte ist das Erreichen des Ausbauziels kaum bzw. nicht erreichbar. Hinzukommen weitere Probleme wie Materialengpässe und Personalmangel.
	Zudem kann ohne zusätzliche Gesetzesänderungen von ca. \num{5} Jahren Planungs- und Bauzeit ausgegangen werden (Planung, Bau und Inbetriebnahme).
	Demnach scheint nicht nur der zeitliche, sondern auch der gesetzliche Rahmen für das Erreichen der Ausbauziele nicht gegeben.
	Folglich werden Reserven aus Kohlekraftwerken auch in ferner Zukunft weiter bestehen müssen, selbst wenn die Wirtschaftlichkeit durch mangelnde Betriebsstunden auch in diesem Fall ungeklärt bleibt.
	
	
	
	