\section{Zusammenfassung und Ausblick}

	1. Fossile Energieträger relativ entspannt-Pegelstände-Sicherheitsbereitschat am Netz-LNG Terminal im Betrieb und Gasspeicher voll-Personalsituation in vielen Fällen auch geregelt siehe ans Netz gegangene Kraftwerke
	2. Stresstest Lücken aufgezeigt (rechtzeitig)-Gegenmaßnahmen eingeleitet
	3. Kapitel 4.4, Ausbauziel, Anreize fehlen \\
	
	
	Aufgrund der genannten Kritikpunkte ist das Erreichen des Ausbauziels kaum bzw. nicht erreichbar. Hinzukommen weitere Probleme wie Materialengpässe und Personalmangel.
	Zudem kann ohne zusätzliche Gesetzesänderungen von ca. fünf Jahren Planungs- und Bauzeit ausgegangen werden (Planung, Bau und Inbetriebnahme).
	Demnach scheint nicht nur der zeitliche und personelle sondern auch der gesetzliche Rahmen für das Erreichen der Ausbauziele nicht vollumfänglich gegeben.
	Folglich werden Reserven aus Kohlekraftwerken auch in ferner Zukunft weiter bestehen müssen, auch wenn die Wirtschaftlichkeit durch mangelnde Betriebsstunden auch in diesem Fall ungeklärt bleibt.
	
	
	
	