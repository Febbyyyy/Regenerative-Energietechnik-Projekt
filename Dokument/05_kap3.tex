\section{Bewertung der Kraftwerksreserven zur Netzstabilisierung und Reserveleistungsvorhaltung}

Im vorherigen Kapitel wurden die verschiedene Reserven in Deutschland beleuchtet und definiert. Außerdem fand eine erste Bewertung der aktuellen Situation statt. Im folgenden sollen nun die technische und logistische Realisierbarkeit bzw. Nutzbarkeit der verschiedenen Reserven beleuchetet werden. Außerdem soll ein Ausblick auf zukünftige Entwicklungen, Möglichkeiten und Gefahren gegeben werden.

Dabei spielt auch der russische Überfall auf die Ukraine eine Rolle. Diese wird in einer Aufnahme der momentanen deutschen Strategie zur Versorgungssicherheit im Winter berücksichtigt.

	\subsection{Bewertung der logistischen Situation der Reserven}
	Die Erhebung der Daten im Punkt Logistik gestaltete sich als herausfordernd. Hier gab es das Problem, dass keine unabhängige Stelle die logistische Situation tatsächlich im Blick hat. Aussagen konnten nur von den Kraftwerksbetreibern selbst getätigt werden.
	
		\subsubsection{Logistischer Stand bei der Braunkohle (Arbeitstitel)} \label{sect: Braunkohle}
		Braunkohlekraftwerke befinden sich in Deutschland ausschließlich in der Sicherheitsbereitschaft. Die zwei Betreiber sind die RWE Power AG mit den zwei Blöcken Niederaußem E und F in Bergheim und dem Block Neurath C in Grevenbroich im Rheinischen Braunkohlerevier und die Lausitzer Energie und Kraftwerke AG, im folgenden LEAG, mit den zwei Blöcken Jänschwalde E und F in Teichland im Lausitzitzer Braunkohlerevier. \\
		
		Die Blöcke des Kraftwerks Jänschwalde werden durch Tagebaue der LEAG in Jänschwalde, Welzow-Süd, Nochten und Reichwalde versorgt. Der Einzugsradius beträgt (hier km Radius). Der Transport wird auf der Schiene durch einen firmeneigenen Eisenbahnbetrieb realisiert. Der Abbau des Tagebaus Jänschwalde stoppt voraussichtlich im Jahr 2023. Dann ist der Tagebau erschöpft. Die Versorgung wird dann von den übrigen drei Abbaustandorten übernommen.
		Da es sich beim Kraftwerk Jänschwalde um ein Großkraftwerk mit 6 Blöcken handelt gibt es kein Personalproblem. 
		(Hier noch etwas zum Wiedereinstieg in den Markt)\\
		
		Die in Sicherheitsbereitschaft befindlichen Blöcke Niederaußem E und F sowie Neurath C werden ebenfalls durch RWE-eigene Tagebaue versorgt. Hierbei handelt es sich um die Förderstätten Garzweiler, Hambach und Inden. Der Einzugsradius beträgt (km) für das Kraftwerk Neurath und (km) für das Kraftwerk Niederaußem. Der Transport erfolgt über eine Eisenbahngesellschaft der RWE. Auch bei diesen drei Blöcken handelt es sich um Teile von größeren Kraftwerkskomplexen. Daher stellt die Personalsituation kein Problem dar.
		
		\subsubsection{Logistischer Stand bei der Steinkohle} \label{sect: Steinkohle}
		Die deutschen Steinkohlekraftwerke, welche nicht mehr aktiv am Markt teilnehmen und noch nicht endgültig stillgelegt sind, befinden sich in Deutschland in der Netzreserve. Diese ist durch das Kohleverstromungsbeendigungsgesetz, nachfolgen KVBG genannt, und Paragraph 13b des Energiewirtschaftsgesetzes, im weiteren Verlauf EnWG, rechtlich verankert. Bei den Betreibern handelt es sich um die EnBW Energie Baden-Würtemberg AG, die STEAG, Uniper Kraftwerke GmbH, Kraftwerk Mehrum GmbH und das Großkraftwerk Mannheim. \\
		
		Die logistische Situation war und ist angespannt. Niedrige Pegelstände von Rhein und Neckar, sowie der fehlende Gleisausbau in Deutschland erschwerten den Steinkohletransport deutlich. Die RAG Deutsche Steinkohle AG war der alleinige Betreiber deutscher Steinkohlebergwerke. Diese stellte den Abbau 2018 komplett ein, da ein wirtschaftlicher Abbau nicht mehr möglich war(Quelle). Damit sind die Betreiber zu \SI{100}{\percent} auf Importe angewiesen.\\
		
		Die EnBW mit den Standorten Heizkraftwerk Altbach/Deizisau HKW 1, Heizkraftwerk Heilbronn HLB 5 und 6 sowie dem Kraftwerk Walheim mit den Blöcken 1 und 2 verstärkte ab Juli 2022 ihre Bemühungen zur Beschaffung, sowie der Erschließung von Flächen zur zusätzlichen Lagerung von Steinkohle.(Pressemeldung und Mail) Auch die Personalsituation ist angespannt, da diese langfristig mit der Premisse der Stillegung geplant wurde. Die fünf Blöcke, welche sich in Netzreserve befinden, werden jedoch aller Voraussicht nach nicht wieder am Markt teilnehmen. Diese können nach eigener Aussage der EnBW aus technischen Gründen nicht ununterbrochen zur Stromerzeugung eingesetzt werden. Grund hierfür ist das fortgeschrittene Alter der Anlagen.\\
		
		Der Stromerzeuger STEAG betreibt zwei Kraftwerke in Netzreserve, die Standorte Boxberg und Weiher 3 im Saarland. Die herausfordernde Lage der Kohleversorgung, auch mit hinblick auf den kommenden Winter, veranlasste das Wirtschaftsministerium des Saarlandes zu einem Logistik-Gipfel. Zu den Teilnehmenden gehörten unter Anderen der Staatssektretär des Bundesministeriums für Digitales und Verkehr, die STEAG selbst und die DB Cargo. Die Gespräche ergaben eine vorrangige Behandlung von Kohletransporten auf der Schiene gegenüber dem öffentlichen Personennahverkehr, im Falle von Versorgungsengpässen. Dies sichert die Belieferung der Kraftwerke mit Brennstoff. Zusätzlich tritt das Kraftwerk Boxberg zum 28.10.2022 wieder in den Markt ein. Das Schwesterkraftwerk Weiher 3 folgt am 31.10.2022. Ermöglicht wird dies durch das Ersatzkraftwerkbereithaltungsgesetz, kurz EKBG, welches eine Rückkehr in den Strommarkt bis Frühjahr 2024 gestattet.\\
		
		Hier Uniper, Mehrum sobald Informationen vorhanden.
		Heyden 4 seit 29.09.2022 wieder am Netz logistik scheint zu stimmen ?
		Mannheim keine Informationen
		
		\subsubsection{Logistischer Stand beim Gas}
		In Folge des russischen Überfalls auf die Ukraine sanken die deutschen Erdgasimporte aus Russland im September nach Stand 03.11.2022 auf null. Zum 03.05.2022 startete die Bundesnetzagentur eine Datenerhebung um den deutschen Gasverbrauch zu ermitteln. In einer Pressemitteilung vom 29.09.2022 wurde von einer notwendigen Einsparung in Höhe von \SI{20}{\percent} gesprochen um die Versorgungssicherheit auch von Erdgaskraftwerken im Winter zu gewährleisten, das Impulspapier von Acatech spricht sogar von 20 bis \SI{30}{\percent}.\\
		
		Deutsche Erdgaskraftwerke, welche nicht mehr im Betrieb sind, befinden sich sowohl in der Netz- als auch in der Kapazitätsreserve. Die Versorgung ist sehr stark von den Netzbetreibern abhängig und die Situation sehr schwer vorhersehbar. Deutschland verfügt zum 04.11.2022 über ein LNG-Terminal zum Import von verflüssigtem Erdgas in Lubmin. Ein weiteres soll im Winter in Betrieb gehen.
		Ein zusätzliches Problem bei der Versorgung besteht in der Ausrichtung der in Europa vorhandenen Gas-Infrastruktur. Dieses ist durch den Aufbau auf einen Gastransport von Ost nach West ausgerichtet. Der Großteil der europäischen LNG-Terminals befindet sich in Westeuropa. Damit ist eine Umkehrung des Gasflusses notwendig. Dieser sogenannte Reverse-Flow wird ermöglicht, indem die Verdichterstationen umgebaut werden. Danach ist ein Gastransport in beide Richtungen bei voller Kapazität möglich.(acatec Impuls)\\
		
		Ein eventuelles Verbot der Verstromung von Gas ist im Notfallplan Gas beschrieben. Dieser ist in drei Teile aufgeteilt. Die Frühwarnstufe und die Alarmstufe lassen einen Eingriff des Gesetzgebers vorerst nicht zu. Er setzt auf marktbasierte Maßnahmen zur Regulierung der verbrauchten Gasmengen. \\
		
		Sollte die Notfallstufe verkündet werden, so behält sich der Staat, in Form von Bundesministerium für Wirtschaft und Klima und Bundesnetzagentur als Lastverteiler, vor, die Substitution von Erdgas durch andere Brennstoffe anzuordnen. Diese ist jedoch nur eine von verschiedenen möglichen Maßnahmen, die getroffen werden könnten. Es steht nicht fest, von welchen Steuermechanismen gebrauch gemacht wird, da die Situation, in der sich Deutschland befindet, eine bisher nie dagewesene ist. 
		
		\subsubsection{Logistischer Stand beim Öl}
		Uniper 3 Ölkraftwerke -> Gespräch mit Herr Zindler 
	
	
	\subsection{Kapazitätssituation für den Winter 2022/23}
	Im ersten Halbjahr 2022 erzeugte Erdgas \SI{30,7} Mrd. {kWh} elektrischen Strom in Deutschland. Dies entspricht einem Anteil von \SI{11,7}{\percent} der Gesamterzeugung. Im Hinblick auf drohende Versorgungsengpässe lohnt sich hier die Betrachtung der Gegenmaßnahmen, die bis hier hin getroffen wurden. \\
	
		\subsubsection{Der zweite Stresstest zum Stromsystem}
		Im Rahmen des vom BMWK in Auftrag gegebenen Stresstests, sollten die Übertragungsnetzbetreiber verschiedene Szenarien für die Versorgungssicherheit im Winter 2022/23 analysieren. Dabei wurden die Gasversorgung, die Steinkohleversorgung und eventuell ausfallende Kapazitäten im Ausland in augenschein genommen.\\
		
		Ein zentraler Punkt der Analyse war die Annahme, dass Polen keinen Strom exportieren kann, da die Versorgung mit Steinkohle durch Lieferengpässe nicht möglich ist. Desweiteren wurde davon ausgegangen, dass Frankreich bis zum Winter nicht alle Kernkraftwerke an den Markt bringen kann. Diese waren auf Grund von zu hohen Temperaturen, Niedrigwasser und Defekte im Sommer teilweise abgeschaltet worden. Außerdem geht die Studie im kritischsten der drei Szenarien davon aus, dass Süddeutschland und Österreich die vertraglich geregelte Redispatchleistung aus Gaskraftwerken nicht liefern können. \\
		
		Das Fazit zeigt, dass in den beiden kritischsten Situationen auch in Deutschland einige Stunden der Lastunterdeckung auftreten. Desweiteren wird gezeigt, dass die deutsche Redispatchleistung in keinem der drei Szenarien ausreicht. Ausländische Leistungen müssen heran gezogen werden. Der Streckbetrieb der deutschen Kernkraftwerke entspannt die Situation. Lastunterdeckungen können weitestgehend vermieden werden, der Bedarf an Redispatch sinkt ebenfalls. Die Empfehlungen der Übertragungsnetzbetreiber stehen auf fünf Säulen. Diese sollen die angespannte Versorgungssituation enspannen:
			\begin{itemize}
				\item Transportkapazitäten erhöhen
				\item Redispatch-Potential im Ausland fokusieren
				\item vertragliches Lastmanagement
				\item Reserven nutzbar machen
				\item Nutzung weiterer Kraftwerkskapazitäten absichern
			\end{itemize}
		
		\subsubsection{Umsetzung der Empfehlungen} \label{sect: Atomausstieg}
		Am 07.10.2022 wurde vom Bundesrat die Novelle zum Energiesicherheitsgesetzt 3.0 verabschiedet. Dieses baut auf verschiedene Säulen zur Anhebung der Kapazitäten der Stromerzeugung. 
			\begin{itemize}
				\item Erhöhung der Stromproduktion aus Photovoltaikanlagen
				\item Anreize für die Verstromung von Biogas
				\item Erhöhung der Produktion von Windstrom
				\item Maßnahmen zur Beschleunigung des Netzausbaus
				\item Maßnahmen im LNG-Beschleunigungsgesetz
				\item Erleichterung für den Brennstoffwechsel
				\item Änderungen im Baugesetztbuch
			\end{itemize}
		
		Zur Erhöhung der Produktionskapazitäten wurde außerdem ein Streckbetrieb für 3 verbeibende deutsche Atomkraftwerke beschlossen. Die Meiler Isar 2, Neckar-Westheim 2 und Emsland werden bis maximal Mitte April am Netz behalten. Diese sollen zusätzlich die Gaskraftwerke entlasten und Netzengpässe abfedern.\\
		Wie in Abschnitt \ref{sect: Steinkohle} beschrieben, konnten Versorgungsengpässe bei der Steinkohle überwunden werden. Außerdem nehmen einzelne Reservekraftwerke seit Oktober wieder aktiv am Markt teil. Insgesamt nehmen Kraftwerke mit einer Gesamtnettokapazität von \SI{2,25}{\giga \watt} wieder am Strommarkt teil. (Tabelle)\\
		Auch die Braunkohlekraftwerke aus der Sicherheitsbereitschaft (siehe \ref{sect: Braunkohle}) nehmen inzwischen wieder aktiv am Markt teil, die Blöcke Niederaußem E, F und Neurath C zum 01.10.2022, Jänschwalde E zum 06.10.2022 und Jänschwalde F zum 15.10.2022. Diese summieren sich zu einer elektrischen Nettoleistung von \SI{1,886}{\mega \watt}. (Tabelle)\\
		
		(Hier noch etwas über den Netzausbau seit September, wenn ich was finde.)
		
		
		
	
	\subsection{Was bedeutet der Atom- und Braunkohleausstieg für die Kraftwerksreserven?}
	In Folge des Reaktorunglücks von Fukushima vom 11. März 2011 beschloss die Bundesregierung am 30.06.2011 den Ausstieg aus der Verstromung von Atomenergie bis zum 31.12.2022. In Folge der in Kapitel \ref{sect: Atomausstieg} erwähnten Laufzeitverlängerung der drei bestehenden Meiler wird dieser voraussichtlich bis 15.04.2023 vollzogen sein. Damit werden \SI{4,29}{\giga \watt} gesicherte Leistung vom Netz gehen.\\
	Weiterhin sollen bis spätestens 2038 alle Kraftwerke, deren Primärenergieträger Kohle ist, stillgelegt werden. Dies Betrifft im näheren die Sicherheitsbereitschaft und Netzreserve mit einer gesicherten Leistung von \SI{6,71}{\giga \watt}.(Kraftwerke die zukünftig in die Reserve überführt werden auch?)\\
	Im Rahmen des Nationalen Energie- und Klimaplans ist eine Reduktion der Treib-
	hausgas-Emissionen im Vergleich zu 1990 um mindestens \SI{55}{\percent} vorgesehen. Dies soll durch Ausbau der Erneuerbaren Energien in Form von Windkraft und Photovoltaik bei gleichzeitiger Reduktion der Endenergieverbräuche in Strom-, Wärme- und Verkehrssektor sowie Abschaltung der fossilen Kohlekraftwerke erreicht werden. Die Endenergiereduktion steht konträr mit dem Ziel der Elektrifizierung des Verkehrs durch die Zulassung von 15 Millionen Elektroautos bis 2030 und dem Ausbau von Elektrolyse-Kapazitäten von \SI{10}{\giga \watt}. (Koalitionsvertrag) Diese sogenannte Mobilitätswende, sowie die Änderung der nationalen Wasserstoffstrategie (BMWK 2020) würde einen Anstieg der Endenergieverbräuche im Stromsektor zur Folge haben.\\
	
	Die Reserven, welche für den Fall einer Dunkelflaute, also kein Wind und keine Sonne, benötigt werden, können die aktuell Verfügbaren Gaskraftwerken, welche von weiten Teilen der Energiewirtschaft als notwendige Brückentechnologie betrachtet werden, nicht decken. Es muss also bis 2038 ein Ausbau der Gaskraftwerksleistung erfolgen, um die Versorgungssicherheit in Deutschland zu gewährleisten. Das Energiewirtschaftliche Institut der Universität Köln spricht in ihrer Analyse des Koalitionsvertrags von einem benötigen Zubau von wasserstofffährigen Gaskraftwerken von \SI{23}{\giga \watt}. (Abbildung EWI Analyse Kraftwerkszusammensetzung 2030)
	Für das Jahr 2030 sieht der Szenariorahmen für den Netzentwicklungsplan der ÜNB eine Jahreshöchstlast von 90-\SI{98}{\giga \watt} vor. Diese müsste notfalls, bei Versorgungsausfall der Erneuerbaren Energien, komplett von den Pumpspeicher-, Biomasse- und Gaskraftwerken erbracht werden.
	
	
	
	
		
	\subsection{Auswirkungen des Ausbaus der erneuerbaren Energien auf die Kraftwerksreserven}
	Netzentwicklungsplan 2030 (2019) bzw. dessen Szenariorahmen
	DENA 2030 Veränderung der Lastflüsse
	Ausbau von Speichermöglichkeiten (Energie- und Wärmespeicher)?
	