\section{Einleitung}

	Kraftwerksreserven sind im deutschen Stromsektor ein weitgefächerter Bereich, welcher einer Erklärung unter verschiedenen Gesichtspunkten bedarf.
	Dieser Begriff vereint mehrere Mechanismen zum Ausgleich von Angebot und Nachfrage am Strommarkt und der zuverlässigen Beständigkeit der Versorgungssicherheit in Deutschland.
	Ziel der folgenden Projektarbeit im Seminarfach "`Regenerative Energietechnik "` ist es die einzelnen Reserven genauer zu beleuchten und anhand unterschiedlicher Einflüsse zu analysieren.
	Im ersten Teil wird der Begriff der Kraftwerksreserve genauer untersucht, da es unterschiedliche Arten von Reserven gibt, um verschiedene Aufgaben zu bewältigen.
	Darüber hinaus wird das Funktionsprinzip des Strommarkts dargestellt, um die Unterschiede zu verdeutlichen.
	Der zweite Teil beschäftigt sich mit der Analyse von Kraftwerksreserven unter gesetzlichen, logistischen sowie wirtschaftlichen Gesichtspunkten.
	Im Anschluss wird Bezug auf die Bedeutung des Atom- und Braunkohleausstiegs für die Kraftwerksreserven genommen.	
	Zuletzt werden positive und negative Auswirkungen durch den Ausbau der erneuerbaren Energien auf die Kraftwerksreserven beleuchtet. \\
	
	Das Thema Versorgungssicherheit rückt nicht zuletzt durch den anhaltenden
	Ukraine Konflikt in den Fokus der Öffentlichkeit.
	Die unregelmäßigen Gaslieferungen aus Russland stellen eine Bedrohung für die deutsche Infrastruktur und Wirtschaft dar.
	Die dadurch verursachte Gasknappheit lässt die Preise für Energie in Deutschland und im Euroraum stark ansteigen.
	Nach Gesetzesänderungen der Bundesregierung rücken z.B. ältere Kohlekraftwerke aus unterschiedlichen Reserven in den Strommarkt nach, um die Verstromung aus Gas zu reduzieren und den Erdgasmarkt und -preis zu entspannen.
	Außerdem setzt der Ausbau der erneuerbaren Energien und die damit einhergehende Volatilität der Stromerzeugung, die Stromnetze vor eine enorme Herausforderung. 
	An dieser Stelle werden weitere Kraftwerksreserven benötigt, um das Netz zu entlasten und regionale Unterschiede in der Erzeugung auszugleichen. 
	Des Weiteren benötigt die Sicherung der Netzfrequenz weitere Ressourcen, um einen Zusammenbruch zu verhindern. 
	Für die schnelle Regelbarkeit der Frequenz stehen vier unterschiedliche Arten von Regelleistungsreserven zur Verfügung. 
	Im Anschluss wird nun auf die beschriebenen Thematiken Bezug genommen und auf die daraus resultierenden Herausforderungen für die Übertragungsnetzbetreiber (ÜNB) und Verteilnetzbetreiber (VNB) eingegangen. 
	
	
	
	
	
	
	\clearpage