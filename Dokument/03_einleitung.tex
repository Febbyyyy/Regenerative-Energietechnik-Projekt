\section{Einleitung}

	Um den Begriff der Kraftwerksreserve besser einzuordnen, folgt eine kurze zeitliche und thematische Einordnung warum die Reserven, gerade aktuell, eine so große Rolle spielen.
	Das Thema Versorgungssicherheit rückt nicht zuletzt durch den anhaltenden Ukraine-Konflikt in den Fokus der Leitmedien.
	Die unregelmäßigen Gaslieferungen aus Russland bedrohen die deutsche Versorgungssicherheit im Hinblick auf Wärme und Strom.
	Darüber hinaus steigen dadurch die Preise für Energie in ungeahnte Höhen.
	Der daraufhin steigende Verkauf an elektrischen Heizlüftern, begründet durch die Angst, dass im Winter kein Gas mehr zur Verfügung steht, kann eine zusätzliche Belastung für das deutsche Stromnetz darstellen.
	Frankreich, welches vorrangig Strom aus Atomkraftwerken bezieht, kann aus Gründen der mangelnden Kühlung und versäumten Wartungsarbeiten aus der Corona Krise, bislang kaum Strom exportieren.
	Hinzu kommt die Energiewende, in welcher die fossile Stromproduktion auf erneuerbare umgestellt werden soll. 
	Die Volatilität von erneuerbaren Energiequellen stellt das Stromnetz sowie die -erzeugung vor enorme Herausforderungen. \\
	
	In der folgenden Projektarbeit im Seminar "`Regenerative Energietechnik"' wird untersucht, inwiefern auftretende Differenzen zwischen Stromangebot- und -nachfrage mit Hilfe der deutschen Kraftwerksreserven ausbalanciert werden können und wo sich diese befinden.
	Im ersten Teil wird der Begriff der Kraftwerksreserve genauer beleuchtet, da es unterschiedliche Arten von Reserven gibt.
	Darüber hinaus wird das Funktionsprinzip des Strommarkts dargestellt, um die Unterschiede der Reserven zu begründen.
	Der zweite Teil beschäftigt sich mit der Bewertung von Kraftwerksreserven unter gesetzlichen, logistischen sowie wirtschaftlichen Gesichtspunkten.
	Im Anschluss wird Bezug auf die Bedeutung des Atom- und Braunkohleausstiegs für die Kraftwerksreserven Bezug genommen.	
	Zuletzt werden positive und negative Auswirkungen auf den Ausbau der erneuerbaren Energien erläutert.
	
	
	
	
	\clearpage